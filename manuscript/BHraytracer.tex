Just putting something here as a placeholder...


%%%%%%%%%%%%%%%%%%%%%%%%%%%%%%%%%%%%%%%%%%%%%%%%%%%%%%%%%%%%%%%%%%%%%

\section{Introduction}
\label{sec:intro}



Motivation... why we are doing this? and why is interesting...


This paper is organized as follows:
in Sec.~\ref{sec:intro} we present the physical problem of tracing rays of light in the presence of a black hole;
in Sec.~\ref{sec:impl} we describe the methods implemented and used to tackle this problem;
in Sec.~\ref{sec:results} we discuss our results and compare to other related works;
and finally in Sec.~\ref{sec:disc} and ~\ref{sec:concl} we present our findings and comment
on possible future extensions to this work.


\begin{itemize}
	\item Background on the kind of black hole
	\item how the differential equation under study works
\end{itemize}


%%%%%%%%%%%%%%%%%%%%%%%%%%%%%%%%%%%%%%%%%%%%%%%%%%%%%%%%%%%%%%%%%%%%%

\section{Implementation}
\label{sec:impl}

\begin{itemize}
	\item how we can model the differential equation in GSL
    	\item how the raytracer works.
    	\item Domain decomposition with MPI (Potentially some images could be larger than system ram, talk about how this is handled)
    	\item Parallel processing with OpenMP
\end{itemize}

%%%%%%%%%%%%%%%%%%%%%%%%%%%%%%%%%%%%%%%%%%%%%%%%%%%%%%%%%%%%%%%%%%%%%

\section{Results}
\label{sec:results}

\begin{itemize}
	\item Some rendered images
    	\item Scaling results after scaling image dimensions, potentially some other scaling analysis 
      which I previously did for my presentation.
\end{itemize}


%%%%%%%%%%%%%%%%%%%%%%%%%%%%%%%%%%%%%%%%%%%%%%%%%%%%%%%%%%%%%%%%%%%%%

\section{Discussion}
\label{sec:disc}

Related and Future work
\begin{itemize}
	\item This technique can be used to raytrace any kind of image under light distortion
    	\item Kerr metric (This was done by some other people before)
\end{itemize}

%%%%%%%%%%%%%%%%%%%%%%%%%%%%%%%%%%%%%%%%%%%%%%%%%%%%%%%%%%%%%%%%%%%%%

\section{Conclusions}
\label{sec:concl}

Summary

%%%%%%%%%%%%%%%%%%%%%%%%%%%%%%%%%%%%%%%%%%%%%%%%%%%%%%%%%%%%%%%%%%%%%
