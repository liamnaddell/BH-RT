%%
%% This is file `sample-manuscript.tex',
%% generated with the docstrip utility.
%%
%% The original source files were:
%%
%% samples.dtx  (with options: `all,proceedings,bibtex,manuscript')
%% 
%% IMPORTANT NOTICE:
%% 
%% For the copyright see the source file.
%% 
%% Any modified versions of this file must be renamed
%% with new filenames distinct from sample-manuscript.tex.
%% 
%% For distribution of the original source see the terms
%% for copying and modification in the file samples.dtx.
%% 
%% This generated file may be distributed as long as the
%% original source files, as listed above, are part of the
%% same distribution. (The sources need not necessarily be
%% in the same archive or directory.)
%%
%%
%% Commands for TeXCount
%TC:macro \cite [option:text,text]
%TC:macro \citep [option:text,text]
%TC:macro \citet [option:text,text]
%TC:envir table 0 1
%TC:envir table* 0 1
%TC:envir tabular [ignore] word
%TC:envir displaymath 0 word
%TC:envir math 0 word
%TC:envir comment 0 0
%%
%%
%% The first command in your LaTeX source must be the \documentclass
%% command.
%%
%% For submission and review of your manuscript please change the
%% command to \documentclass[manuscript, screen, review]{acmart}.
%%
%% When submitting camera ready or to TAPS, please change the command
%% to \documentclass[sigconf]{acmart} or whichever template is required
%% for your publication.
%%
%%
%\documentclass[manuscript,screen,review]{acmart}
%\documentclass[acmsmall]{acmart}
\documentclass[acmsmall]{acmart}

%%
%% \BibTeX command to typeset BibTeX logo in the docs
\AtBeginDocument{%
  \providecommand\BibTeX{{%
    Bib\TeX}}}

%% Rights management information.  This information is sent to you
%% when you complete the rights form.  These commands have SAMPLE
%% values in them; it is your responsibility as an author to replace
%% the commands and values with those provided to you when you
%% complete the rights form.
\setcopyright{acmlicensed}
\copyrightyear{2025}
\acmYear{2025}
\acmDOI{XXXXXXX.XXXXXXX}

%% These commands are for a PROCEEDINGS abstract or paper.
\acmConference[PEARC '25]{PEARC'25: The Power of Collaboration}{July 20--24, 2025}{Columbus, Ohio}
%%
%%  Uncomment \acmBooktitle if the title of the proceedings is different
%%  from ``Proceedings of ...''!
%%
%%\acmBooktitle{Woodstock '18: ACM Symposium on Neural Gaze Detection,
%%  June 03--05, 2018, Woodstock, NY}
\acmISBN{978-1-4503-XXXX-X/18/06}


%%
%% Submission ID.
%% Use this when submitting an article to a sponsored event. You'll
%% receive a unique submission ID from the organizers
%% of the event, and this ID should be used as the parameter to this command.
%%\acmSubmissionID{123-A56-BU3}

%%
%% For managing citations, it is recommended to use bibliography
%% files in BibTeX format.
%%
%% You can then either use BibTeX with the ACM-Reference-Format style,
%% or BibLaTeX with the acmnumeric or acmauthoryear sytles, that include
%% support for advanced citation of software artefact from the
%% biblatex-software package, also separately available on CTAN.
%%
%% Look at the sample-*-biblatex.tex files for templates showcasing
%% the biblatex styles.
%%

%%
%% The majority of ACM publications use numbered citations and
%% references.  The command \citestyle{authoryear} switches to the
%% "author year" style.
%%
%% If you are preparing content for an event
%% sponsored by ACM SIGGRAPH, you must use the "author year" style of
%% citations and references.
%% Uncommenting
%% the next command will enable that style.
%%\citestyle{acmauthoryear}



%%% ADDITIONAL PACKAGES


%\usepackage{markdown}
%\markdownSetup{
%  renderers = {
%    link     = {#1},        % Render a link as the link label.
%    emphasis = {\emph{#1}}, % Render emphasis using `\emph`.
%  }
%}    

%%%%%%%%%%%%%%%%%%%%%%%%


%%
%% end of the preamble, start of the body of the document source.
\begin{document}

%%
%% The "title" command has an optional parameter,
%% allowing the author to define a "short title" to be used in page headers.
\title[BH RayTracer]{Parallel Ray Tracing of Black Hole Images Using the Schwartzschield Metric}

%%
%% The "author" command and its associated commands are used to define
%% the authors and their affiliations.
%% Of note is the shared affiliation of the first two authors, and the
%% "authornote" and "authornotemark" commands
%% used to denote shared contribution to the research.
\author{Liam Nadell}
\email{liam.naddell@mail.utoronto.ca}
%\orcid{1234-5678-9012}
\author{Marcelo Ponce}
\email{m.ponce@utoronto.ca}
\affiliation{%
  \institution{Department of Computer and Mathematical Sciences, University of Toronto Scarborough}
  \city{Toronto}
  \state{ON}
  \country{Canada}
}


%%
%% By default, the full list of authors will be used in the page
%% headers. Often, this list is too long, and will overlap
%% other information printed in the page headers. This command allows
%% the author to define a more concise list
%% of authors' names for this purpose.
%\renewcommand{\shortauthors}{Trovato et al.}

%%
%% The abstract is a short summary of the work to be presented in the
%% article.
\begin{abstract}
Rendering images of black holes by utilizing raytracing techniques is a common
methodology employed in many aspects of scientific and astrophysical visualizations.
Similarly general raytracing techniques are widely used in areas related of computer graphics.
In this work we describe the implementation of a parallel open-source code that can raytrace images in the presence of a black hole geometry.
We do this by combining a couple of different techniques usually present in parallel scientific computing,
such as, mathematical approximations, utilization of scientific libraries, shared-memory and distributed-memory parallelism.

%\CMT{move to intro?}
%The path a light ray takes in orbit of a simple black hole can be modeled by the partial differential equation $u^\prime\prime - u = 3Mu^2$, which relates the Euclidean distance and azimuthal angle between the light ray and the black hole center, to the mass of the black hole. Using GSL and standard raytracing methods, we can trace the path of a light ray through 3D space, and determine the interactions that light ray may develop in a 3D scene. This comes at the cost of significant computational complexity, which is handled by decomposing the problem with MPI, and rendering distinct rays in parallel. TODO: Mention educational section

\end{abstract}

%%
%% The code below is generated by the tool at http://dl.acm.org/ccs.cfm.
%% Please copy and paste the code instead of the example below.
%%
\begin{CCSXML}
<ccs2012>
   <concept>
       <concept_id>10010405.10010432.10010441</concept_id>
       <concept_desc>Applied computing~Physics</concept_desc>
       <concept_significance>500</concept_significance>
       </concept>
   <concept>
       <concept_id>10010405.10010432.10010435</concept_id>
       <concept_desc>Applied computing~Astronomy</concept_desc>
       <concept_significance>500</concept_significance>
       </concept>
   <concept>
       <concept_id>10010147.10010169.10010170.10010171</concept_id>
       <concept_desc>Computing methodologies~Shared memory algorithms</concept_desc>
       <concept_significance>500</concept_significance>
       </concept>
   <concept>
       <concept_id>10010147.10010919.10010177</concept_id>
       <concept_desc>Computing methodologies~Distributed programming languages</concept_desc>
       <concept_significance>300</concept_significance>
       </concept>
 </ccs2012>
\end{CCSXML}

\ccsdesc[500]{Applied computing~Physics}
\ccsdesc[500]{Applied computing~Astronomy}
\ccsdesc[500]{Computing methodologies~Shared memory algorithms}
\ccsdesc[500]{Computing methodologies~Distributed programming languages}



%%
%% Keywords. The author(s) should pick words that accurately describe
%% the work being presented. Separate the keywords with commas.
\keywords{Black holes, raytracing, parallel programming, mathematical approximations, scientific libraries.}


%\received{20 February 2007}
%\received[revised]{12 March 2009}
%\received[accepted]{5 June 2009}


%%
%% This command processes the author and affiliation and title
%% information and builds the first part of the formatted document.
\maketitle






Just putting something here as a placeholder...

\begin{verbatim}
Outline:
1. Abstract
2. Introduction
   * Background on the kind of black hole
    * how the differential equation under study works
3. Implementation
    * how we can model the differential equation in GSL
    * how the raytracer works.
    * Domain decomposition with MPI (Potentially some images could be larger than system ram, talk about how this is handled)
    * Parallel processing with OpenMP
4. Results
    * Some rendered images
    * Scaling results after scaling image dimensions, potentially some other scaling analysis 
      which I previously did for my presentation.
5. Related and Future work
    * This technique can be used to raytrace any kind of image under light distortion
    * Kerr metric (This was done by some other people before)
6. Summary

\end{verbatim}






%%
%% The next two lines define the bibliography style to be used, and
%% the bibliography file.
\bibliographystyle{ACM-Reference-Format}
\bibliography{refs}


%%
%% If your work has an appendix, this is the place to put it.
\begin{comment}
\appendix

\section{Research Methods}

\subsection{Part One}

Lorem ipsum dolor sit amet, consectetur adipiscing elit. Morbi
malesuada, quam in pulvinar varius, metus nunc fermentum urna, id
sollicitudin purus odio sit amet enim. Aliquam ullamcorper eu ipsum
vel mollis. Curabitur quis dictum nisl. Phasellus vel semper risus, et
lacinia dolor. Integer ultricies commodo sem nec semper.

\subsection{Part Two}

Etiam commodo feugiat nisl pulvinar pellentesque. Etiam auctor sodales
ligula, non varius nibh pulvinar semper. Suspendisse nec lectus non
ipsum convallis congue hendrerit vitae sapien. Donec at laoreet
eros. Vivamus non purus placerat, scelerisque diam eu, cursus
ante. Etiam aliquam tortor auctor efficitur mattis.

\section{Online Resources}

Nam id fermentum dui. Suspendisse sagittis tortor a nulla mollis, in
pulvinar ex pretium. Sed interdum orci quis metus euismod, et sagittis
enim maximus. Vestibulum gravida massa ut felis suscipit
congue. Quisque mattis elit a risus ultrices commodo venenatis eget
dui. Etiam sagittis eleifend elementum.

Nam interdum magna at lectus dignissim, ac dignissim lorem
rhoncus. Maecenas eu arcu ac neque placerat aliquam. Nunc pulvinar
massa et mattis lacinia.
\end{comment}

\end{document}
\endinput
%%
%% End of file `sample-manuscript.tex'.
